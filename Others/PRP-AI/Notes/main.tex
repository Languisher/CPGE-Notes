\documentclass[11pt, a4paper]{article}

% Set the title of the current document to be produced.
\newcommand{\doctitle}{Neural Networks}
% Command for the due date of the homework.
\newcommand{\duedate}{\color{rltred}{\faCalendarCheckO { }Due date: May 1st, before midnight \faCalendarCheckO	}}

%------------------------------------------------------------
% Import commands for both teacher and course information.  | 
% NOTE: Change your teacher and course info in these files. |
%------>------>------>------>------>------>------>------>-->|
\newcommand{\instructor}{Lin Nan} 
\newcommand{\college}{Shanghai Jiao Tong University}
\newcommand{\semester}{Summer 2022}
\newcommand{\coursetitle}{Neural Networks} 
\newcommand{\coursenumber}{PRP}                              %|   
%
%------------------------------------------------------------
%-- Import packages and custom command definitons.          |
%------>------>------>------>------>------>------>------>-->|
\input{includes/packages}                                  %|  
\input{includes/custom-commands}   
%
%---> Genereate & inject metadata describing                |
%     the produced document                                 |
%--------------------------------------------------------------
%-- Set up the hyperref package.                              |
%-- Generate and inject metadate in the produced PDF document |
%------>------>------>------>------>------>------>------>-->---
 \hypersetup{pdfauthor={\instructor},%
    pdftitle={\coursetitle},%
    pdfsubject={\doctitle},%
    pdfkeywords={\college},%
    pdfproducer={LaTeX},%
    pdfcreator={pdfLaTeX},
    bookmarks,
    bookmarksnumbered = true,
    bookmarksopen     = true,
    pdfpagelabels     = true,
    pdfstartview={XYZ null null 1.2}
}                                  %|
%------------------------------------------------------------

\topmargin      -60pt

\begin{document} 
    
%-------------------------------------------------------------
%-- Make the header of the document                          |
%------>------>------>------>------>------>------>------>--> |
%--------------------------------------------------------------------------
%- The following produces the document header including the title.        |
%- The document header includes: the college/university name, faculty,    |
%  department, course number and title as well as the assignment/homework | 
%  title and due date.                                                    | 
%-------------------------------------------------------------------------|
%
\noindent % <-- need to have this first.
%
\begin{minipage}{.40\textwidth}
    {\color{darkred} \faSchool} { \textsc{\college}}{ } {\color{darkred} \faSchool}\\ 
    \small\textsc{ \semester}
\end{minipage}%
\hfill	
\begin{minipage}{0.60\textwidth}%
    \raggedleft%
    {\Large \textsc{\coursetitle}\par}
    \doublerule % insert a double rule.
    \textsc{Author}: \instructor\\
\end{minipage}%
\vspace{2.8cm}
{
    %--> Insert homework title and due date.
    \hrule\vspace{.2cm}
    \centering
    {\scshape 
        \Large \color{darkestblue}{\doctitle}\par}
    \vspace{.3cm}    
}

\vspace{1.2cm}


\vskip .3in
%
\tableofcontents

\clearpage

%-------------------------------------------------------------
%-- Begin here!                          |
%------>------>------>------>------>------>------>------>--> |

\section{Bases}
\subsection{Architectures}
\begin{figure}[H] %h:当前位置, t:顶部, b:底部, p:浮动页
    \centering
    \includegraphics[width=\textwidth]{./fig/neural-networks-architectures.png}
    \caption{neural-networks-architectures}
    \label{fig:neural-networks-architectures}
\end{figure}

We will focus on one node, by noting:

\begin{itemize}
    \item $i$ : the  $i$-th layer of the network
    \item $j$ : the $j$-th hidden units of the layer
    \item $(\vec{x},y)$ : datasets, where $\vec{x}$ is the input and $y$ is the desired output. $\vec{x}\in \mathcal{R}^{n_x}$ has $n_x$ variables
    \item $\vec{w}\in \mathcal{R}^{n_x}$ : weight, each $w$ corresponds to one $x$
    \item $b\in \mathcal{R}$ :  bias
    \item $z$ : output
\end{itemize}

We have:

\begin{tcolorbox}
    \textbf{Forward propagation} (before using the activation function):
   \[
       z_j^{[i]} = (\vec{w}_j^{[i]})^{T} \cdot \vec{x} + b_j^{[i]}
   \]
\end{tcolorbox}

And then, \textbf{activation functions} $\hat{y}=g(z)$ are used at the end of a hidden unit to introduce non-linear complexities to the model.

\begin{figure}[H] %h:当前位置, t:顶部, b:底部, p:浮动页
    \centering
    \includegraphics[width=\textwidth]{./fig/activation-functions.png}
    \caption{activation-functions}
    \label{fig:activation-functions}
\end{figure}

Suming up:

\begin{tcolorbox}
    We have dataset $(\vec{x},y)$. With input $x\in \mathcal{R}^{n_x}$, and the help of $\vec{w},b$ and $g$,
\[
    \vec{x} \mapsto z \mapsto \hat{y} = g(z)
\]
    In this way having $\hat{y}$ in our model and $y$ in reality.
\end{tcolorbox}
\subsection{Loss functions}
Here, we are going to measure the difference between $y$ and $\hat{y}$. \textbf{Loss functions} are to measure how well our algorithm is doing based on one single sample.
\begin{tcolorbox}
    \textbf{Cross-entropy loss}:
    \[
    L(\hat{y},y) = -[y\log \hat{y}+(1-y)\log (1-\hat{y})]
    \]
\end{tcolorbox}
\begin{tcolorbox}
    \textbf{Mean squared error loss}:
    \[
    L(\hat{y},y) = (y - \hat{y})^{2}
    \]
\end{tcolorbox}
Also, we have the \textbf{cost function} based on a number of samples:

\begin{tcolorbox}
    \textbf{Cost function}: to measure how well you're doing in the entire training set.

    Here, $n$ means the $n$-th sample, and $m$ is the total number of the samples.
    \[
        J(w,b) = \frac{1}{m} \sum_{n=1}^{m} L(\hat{y}^{[n]}, y ^{[n]})
    \]
\end{tcolorbox}

\subsection{Gradient Descent}

\textbf{Gradient Descent} is based on a convex function and tweaks its parameters iteratively to minimize a given function (here, the cost function) to its local minimum.
\begin{figure}[H]
\begin{minipage}[t]{0.5\linewidth} 
\centering 
\includegraphics[width=2.0in]{./fig/gradient-descent-1D.png} 
\caption{gradient-descent-1D} 
\label{fig:gradient-descent-1D} 
\end{minipage}% 
\begin{minipage}[t]{0.5\linewidth} 
\centering 
\includegraphics[width=2.5in]{./fig/gradient-descent-3D.png} 
\caption{gradient-descent-3D} 
\label{fig:gradient-descent-3D} 
\end{minipage} 
\end{figure}

We note :
\begin{itemize}
    \item $a$ : current position
    \item $b$ : the next position
    \item $\alpha$ : a waiting factor
    \item $\nabla  f(a)$ : the direction of the steepest descent at $a$
\end{itemize}
\begin{tcolorbox}
    What \textbf{gradient descent} does:
    \[
        b = a - \alpha \cdot \nabla f(a)
    \]
\end{tcolorbox}

We define $\alpha$ as the \textbf{learning rate}, which indicates at which space the weights get updated.

It is very important for us to select a appropriate value of $\alpha$.
\begin{figure}[H] %h:当前位置, t:顶部, b:底部, p:浮动页
    \centering
    \includegraphics[width=0.5\textwidth]{./fig/gradient-descent-learning-rate.png}
    \caption{gradient-descent-learning-rate}
    \label{fig:gradient-descent-learning-rate}
\end{figure}

After several iterations by repeating the method until $|\nabla f|< \varepsilon$, which means it has converged, then stop.

\subsection{Backpropogation}

\textbf{Backpropogation} is an algorithm for supervised learning of artificial neural networks using \textbf{gradient descent}.

Here, we have :
\[
\hat{y} = g(z), z = w^{T}x+b
\]

For one single node with $(x\in \mathcal{R}^{n_x},y)$, we focus on one variable $w_k$ which correspond to $x_k \in \{x_1, \ldots,x_{n_x}\}$:
\begin{align*}
    \nabla_w = \frac{\partial L(\hat{y},y)}{\partial w_k}
    &= \frac{\partial L(\hat{y},y)}{\partial \hat{y}} \cdot \frac{\text{d}\hat{y}}{\text{d}z} \cdot \frac{\partial z}{\partial w_k} \\
    &= \frac{\partial L(\hat{y},y)}{\partial z} \cdot x_k 
\end{align*}

If $g$ is the sigmoid function, and the loss function is the cross-entropy loss function, then for each $w_k$,
\[
\nabla_w =\frac{\partial L(\hat{y},y)}{\partial w_k} = (\hat{y}-y) \cdot x_k
\]
\[ \begin{cases}
    w_k &:= w_k - \alpha(\hat{y}-y)x_k \\
    b_k &:= b_k - \alpha(\hat{y}-y)
\end{cases}
\]

\begin{tcolorbox}
$w$ and $b$ are updated as follows :
\begin{enumerate}
    \item Take a batch of training data
    \item Perform \textbf{forward propagation} to obtain the corresponding loss
    \item \textbf{Backpropagate} the loss to get the gradients
    \item Use the gradients to update the weights of the network
\end{enumerate}
\end{tcolorbox}
The whole process is:
\begin{verbatim}
    ---Initializing---
    g = (an activation function)
    J = 0
    L = (Cross-entropy loss)
    For p = 1 to (the total number of inputs)
        dw_p = 0
        w_p = (random but appropriate number)
    db = 0
    alpha = (learning rate)

    ---For all the samples---
    For n = 1 to m
        ---Forward Propagation---
        z = wx + b
        a = g(z)
        J += L(a,y)
        ---Backpropagation---
        dz = (after calculation)
        For p = 1 to (the total number of inputs)
            dw_p += x_p * dz
        db += dz
    
    ---Update w and b by using the gradients---
    J /= m
    For p = 1 to (the total number of inputs)
        dw_p /= m
        w_p := w_p - alpha * dw_p
    b := b - alpha * db
\end{verbatim}

\end{document}