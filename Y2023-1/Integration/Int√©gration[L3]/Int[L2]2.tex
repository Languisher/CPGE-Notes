\chapter{Intégrales multiples} % (fold)
\label{chap:Intégrales multiples}

\section{Intégrales doubles} % (fold)

% subsection  (end)
\subsection{Théorème de Fubini} % (fold)
\label{sub:Théorème de Fubini}

Une région \textbf{admissible}. définie à l'aide de deux fonctions d'une variable 
\begin{equation}
  U = \{(x, y) \in \mathbb{R}^{2}, \;a \le x \le b, \;\varphi(x) \le y \le \psi(x)\}
\end{equation}
ou
\begin{equation}
  U = \{(x, y) \in \mathbb{R}^{2}, \;c \le y \le d, \;\alpha(y) \le x \le \beta(y)\}
\end{equation}
\begin{Theorem}{
    Théorème de Fubini
  }{}
Si $f$ est une fonction continue sur un domaine admissible, on calcule l'\textbf{intégrale double} de $f$ sur $U$ en calculant deux intégrales simples : 
\begin{equation}
  \iint _{U} f(x,y) \mathrm{d} x \mathrm{d}y = \int_{a}^{b} \left[\int_{\varphi(x)}^{\psi(x)} f(x,y) \mathrm{d}y\right] \mathrm{d} x = \int_{c}^{d} \left[ \int_{\alpha(y)}^{\beta(y)}f(x,y) \mathrm{d}x\right] \mathrm{d}y
\end{equation}
\end{Theorem}

\begin{Theorem}{Propriétés de l'intégrale double}{}
\begin{itemize}

    \item Linéarité 
    \item Additivité
    \item Positivité

\end{itemize}
\end{Theorem}

\subsection{Changement de variables} % (fold)
\label{sub:Changement de variables}

% subsection Changement de variables (end)
\begin{Theorem}{Changement de variables}{}

\end{Theorem}

\subsection{Aire d'un domaine plan} % (fold)
\label{sub:Aire d'un domaine plan}
\begin{Definition}[colbacktitle=red!75!black]{Aire d'un domaine plan}{}
On définit l'aire d'un domaine admissible $D \subset \mathbb{R} ^{2}$ par 
\begin{equation}
  \mathcal{A} (D) = \iint _{D} 1 \mathrm{d}x \mathrm{d}y
\end{equation}

\end{Definition}





% subsection Aire d'un domaine plan (end)


% subsection Théorème de Fubini (end)
% chapter Intégrales multiples (end)
