\documentclass[11pt, a4paper]{article}

% Set the title of the current document to be produced.
\newcommand{\doctitle}{Linear Algebra}
% Command for the due date of the homework.
\newcommand{\duedate}{\color{rltred}{\faCalendarCheckO { }Due date: May 1st, before midnight \faCalendarCheckO	}}

%------------------------------------------------------------
% Import commands for both teacher and course information.  |
% NOTE: Change your teacher and course info in these files. |
%------>------>------>------>------>------>------>------>-->|
\newcommand{\instructor}{Lin Nan}
\newcommand{\college}{Shanghai Jiao Tong University}
\newcommand{\semester}{Autumn 2022}
\newcommand{\coursetitle}{Linear Algebra}
\newcommand{\coursenumber}{MIT18.06}                              %|
%
%------------------------------------------------------------
%-- Import packages and custom command definitons.          |
%------>------>------>------>------>------>------>------>-->|
\input{includes/packages}                                  %|
\input{includes/custom-commands}
%
%---> Genereate & inject metadata describing                |
%     the produced document                                 |
%--------------------------------------------------------------
%-- Set up the hyperref package.                              |
%-- Generate and inject metadate in the produced PDF document |
%------>------>------>------>------>------>------>------>-->---
 \hypersetup{pdfauthor={\instructor},%
    pdftitle={\coursetitle},%
    pdfsubject={\doctitle},%
    pdfkeywords={\college},%
    pdfproducer={LaTeX},%
    pdfcreator={pdfLaTeX},
    bookmarks,
    bookmarksnumbered = true,
    bookmarksopen     = true,
    pdfpagelabels     = true,
    pdfstartview={XYZ null null 1.2}
}                                  %|
%------------------------------------------------------------

\topmargin      -60pt

\begin{document}

%-------------------------------------------------------------
%-- Make the header of the document                          |
%------>------>------>------>------>------>------>------>--> |
%--------------------------------------------------------------------------
%- The following produces the document header including the title.        |
%- The document header includes: the college/university name, faculty,    |
%  department, course number and title as well as the assignment/homework | 
%  title and due date.                                                    | 
%-------------------------------------------------------------------------|
%
\noindent % <-- need to have this first.
%
\begin{minipage}{.40\textwidth}
    {\color{darkred} \faSchool} { \textsc{\college}}{ } {\color{darkred} \faSchool}\\ 
    \small\textsc{ \semester}
\end{minipage}%
\hfill	
\begin{minipage}{0.60\textwidth}%
    \raggedleft%
    {\Large \textsc{\coursetitle}\par}
    \doublerule % insert a double rule.
    \textsc{Author}: \instructor\\
\end{minipage}%
\vspace{2.8cm}
{
    %--> Insert homework title and due date.
    \hrule\vspace{.2cm}
    \centering
    {\scshape 
        \Large \color{darkestblue}{\doctitle}\par}
    \vspace{.3cm}    
}

\vspace{1.2cm}


\vskip .3in
%
\tableofcontents

\clearpage

%-------------------------------------------------------------
%-- Begin here!                          |
%------>------>------>------>------>------>------>------>--> |

\section{$Ax=b$ and the four subspaces}

\subsection{Vectors}

\textbf{Column vector} $\boldsymbol{v} = (v_1, \ldots, v_n)$  : ($n$-dimensional vector)
\[
\boldsymbol{v} = \begin{bmatrix} v_1\\ \vdots\\ v_n \end{bmatrix}
\]

\textbf{Linear combination} : vector addition + scalar multiplication
\[
c\boldsymbol{v} + d\boldsymbol{w}
\]

\begin{itemize}
    \item dot product : $\boldsymbol{v} \cdot \boldsymbol{w} =v_1w_1+\ldots+v_nw_n$\\ 
    When dot product is 0, they are perpendicular vectors

    \item length : $ \mid \boldsymbol{v} \mid = \sqrt{\boldsymbol{v} \cdot \boldsymbol{v} } = \sqrt{v_1^{2}+v_2^{2}+\ldots+v_n^{2}}$ \\
    A unit vector $\boldsymbol{u} $ is a vector whose length equals 1 \\
    $\boldsymbol{u} =\boldsymbol{v} /  \mid \boldsymbol{v}  \mid $ is a unit vector in the same direction as $\boldsymbol{v} $
\end{itemize}

%-------------------------------------------------------------
%-- Begin here!                          |
%------>------>------>------>------>------>------>------>--> |
\end{document}