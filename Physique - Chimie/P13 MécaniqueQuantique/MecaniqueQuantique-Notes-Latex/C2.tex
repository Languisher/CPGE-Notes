\chapter{Représentation en position et en impulsion, Opérateurs}

\begin{tcolorbox}
  \begin{itemize}

    \item Représentation en impulsion
    \begin{itemize}

        \item Fonction d’onde en représentation en position et en impulsion, la transformation entre les deux 
        \item Exemple : état localisé, onde de de Broglie 
        \item La valeur moyenne de la position et impulsion dans la représentation en position et en impulsion, l’opérateur impulsion 
        \item La valeur moyenne d’impulsion par la notation de Dirac 
        \item Commutateur de la position et l’impulsion
    \end{itemize}
   \item Principe de correspondance et exemples

  \end{itemize}
\end{tcolorbox}
\section{Transformée de Fourier}
\subsection{Transformée de Fourier en impulsion}
\label{Transformée de Fourier}

\begin{Definition}[colbacktitle=red!75!black]{Transformée de Fourier en implusion}{}
Pour une fonction de la position $f(x)$, on peut écrire $g(p)$ qui est appelée \textbf{Transformée de Fourier en impulsion} de la fonction $f(x)$ :
  \[
    \boxed{f(x) = \frac{1}{\sqrt{2 \pi \hbar}} \int_{- \infty} ^{+ \infty} g(p) e ^{ipx / \hbar} \mathrm{d}p, \; g(p) = \frac{1}{\sqrt{2 \pi \hbar}} \int_{- \infty} ^{+ \infty} f(x)e ^{{\color{red} - }ipx / \hbar} \mathrm{d}x}
  \]

  D'un point de vue physique, $f(x)$ signifie que $f(x)$ est considérée comme une combinaison linéaire continue de \underline{fonctions sinusoïdales} de $x$, de période $\lambda = h/p$.
\end{Definition}

\subsection{Égalité de Parseval-Plancherel} % (fold)
\label{sub:Égalité de Parseval-Plancherel}

% subsection Égalité de Parseval-Plancherel (end)
\begin{Prop}{Égalité de Parseval-Plancherel}{}
Soit $f_1(x)$, $f_2(x)$. Soit 
\[
  f_1(x) \underset{T.F.}{\iff} g_1(p), \quad f_2(x) \underset{T.F.}{\iff} g_2(p)
\]

Nous avons 
\begin{equation}
  \boxed{\displaystyle\int_{- \infty}^{+ \infty} f_1 ^{*}(x) f_2(x) \mathrm{d} x = \displaystyle\int_{- \infty}^{+ \infty} g_1 ^{*}(p) g_2(p) \mathrm{d} p}
  \label{eq:Parseval-Plancherel}
\end{equation}
En particulier, en prenant $f_1(x) = f_2(x) = f(x)$ donc $g_1(p) = g_2(p) = g(p)$, on a 
\[
  \int_{- \infty}^{ + \infty} |f(x)| ^{2} \mathrm{d} x = \int_{- \infty}^{ + \infty} |g(p)| ^{2} \mathrm{d} p
\]
\end{Prop}
 
\subsection{Transformée de Fourier et dérivation} % (fold)

% subsection  (end)
\begin{Prop}{Transformée de Fourier et dérivation}{}
Si $f(x)$. 
\begin{equation}
  \left[ f(x) \underset{T.F.}{\iff} g(p) \right] \implies \left[  f_n(x) = \frac{\partial ^{(n) }f(x)}{\partial x ^{(n)}} \underset{T.F}{\iff} g_n(p) = \left( \frac{ip}{\hbar}  \right) ^{n} g(p)
 \right] \label{eq:2}
\end{equation}
\end{Prop}

\begin{myproof}{}{}
Soit $f_1(x) = f'(x)$, donc 

\begin{align*}
  g_1(p) &= \frac{1}{\sqrt{2 \pi \hbar}}  \displaystyle\int_{- \infty}^{+\infty} f'(x) \exp(-i \frac{p x}{\hbar} ) \mathrm{d}x \\
         &= \frac{1}{\sqrt{2 \pi \hbar}}  \displaystyle\int_{- \infty}^{+\infty}  \exp(-i \frac{p x}{\hbar} ) \mathrm{d}f \\
         &= \frac{1}{\sqrt{2 \pi \hbar}}  \left[ f \exp\left( - i \frac{p x}{\hbar} \right)| _{- \infty} ^{+ \infty} - \left( - \frac{ip}{\hbar}  \right) \int_{- \infty}^{+\infty} f \exp \left( - i \frac{p x}{\hbar}  \right)\mathrm{d}x\right] \\ 
         &= 0 +  \frac{i p }{\hbar}  \frac{1}{\sqrt{2 \pi \hbar}} \int_{- \infty}^{+\infty} f(x) \exp \left( -i \frac{p x}{\hbar}  \right) \mathrm{d} x \\ 
         &= \frac{ip}{\hbar}  g(p)
\end{align*}

Récurrence.
\end{myproof}

\subsection{Exemples} % (fold)
\label{sub:Exemples}

% subsection Exemples (end)
\begin{Prop}{Transformée de Fourier de la distribution de Dirac}{}
\[
  f(x) = \delta(x- x_0) \underset{T.F.}{\iff} g(p)  = \frac{1}{\sqrt{2 \pi \hbar}}  \exp \left( -i \frac{p x_0}{\hbar}  \right)
\]
On a $\langle x \rangle = x_0$, $\Delta x = 0$, $\Delta p \to \infty$
\end{Prop}

\begin{myproof}{}{}
\[
  g(p) = \frac{1}{\sqrt{2 \pi \hbar}}  \displaystyle\int_{- \infty}^{+ \infty} \delta(x-x_0) \exp \left( -i \frac{px}{\hbar}  \right) \mathrm{d} x = \frac{1}{\sqrt{2 \pi \hbar}}  \exp \left( -i \frac{p x_0}{\hbar}  \right)
\]
\end{myproof}

\begin{Prop}{Transformée de Fourier de l'onde de de Broglie}{}
\[
  f(x) = A e^{ip_0 x / \hbar} e^{ - iEt/\hbar} \underset{T.F.}{\iff} g(p) = \frac{A e^{-iEt/\hbar}}{\sqrt{2 \pi \hbar}} 2 \pi \hbar \delta(p-p_0) \propto \delta(p-p_0)
\]
On a $\Delta x \to \infty$, $\langle p \rangle = p_0$, $\Delta p =0$
\end{Prop}

\newpage
\section{Fonction d'onde en implusion} % (fold)
\label{sec:Fonction d'onde en implusion}

\subsection{Définition} % (fold)

% subsection Définition (end)

\begin{Definition}[colbacktitle=red!75!black]{\textbf{Représentation de la fonction d'onde en impulsion et en position}}{}
\textbf{Représentation de la fonction d'onde en impulsion} : $\varphi(p,t)$

\textbf{Représentation de la fonction d'onde en position} : $\psi(x,t)$
\begin{itemize}

    \item Cas 1D
\begin{gather}
  \psi (x, t) = \frac{1}{\sqrt{2 \pi \hbar}}  \int_{ - \infty}^{ + \infty} \varphi(p,t) e ^{ipx / \hbar} \mathrm{d} p, \quad
  \varphi (p, t) = \frac{1}{\sqrt{2 \pi \hbar}}  \int_{ - \infty}^{ + \infty} \psi(x,t) e ^{-ipx / \hbar} \mathrm{d} x
\end{gather}

\item Cas 3D 
  \begin{gather}
    \psi(M,t) = \frac{1}{(2 \pi \hbar) ^{3 / 2}}   \iiint \varphi(\overrightarrow{p},t) e ^{\frac{i}{\hbar} \overrightarrow{p}. \overrightarrow{OM}} \mathrm{d} p_x \mathrm{d} p_y \mathrm{d} p_z, \quad 
    \varphi(\overrightarrow{p},t) = \frac{1}{(2 \pi \hbar) ^{3 / 2}}   \iiint \psi(M,t) e ^{-\frac{i}{\hbar} \overrightarrow{p}. \overrightarrow{OM}} \mathrm{d} V_M
  \end{gather} 

\end{itemize}
\end{Definition}

\subsection{Mesure de l'impulsion} % (fold)
\label{sub:Mesure de l'impulsion}

La \underline{probabilité} de mesure de l'impulsion de la particule dans l'intervalle $[p, p + \mathrm{d}p]$ est donnée par : 
\begin{equation}
  \mathrm{d}P _{p, t, \mathrm{d}p} = |\varphi(p, t)| ^{2} \mathrm{d}p
\end{equation}

Justifications : 
\begin{itemize}

    \item D'après l'égalité de Parseval-Plancherel, $\varphi(p,t)$ est normalisée : 
      \begin{equation}
        \int_{- \infty}^{+ \infty} |\varphi(p,t)| ^{2} \mathrm{d} p = 1
      \end{equation}

    \item Pour une onde de de Broglie, 
      \begin{equation}
        \varphi(p,t) \propto \delta(p - p_0)
      \end{equation}

\end{itemize}

De même façon, dans le cas tridimensionnel :
\begin{equation}
  \mathrm{d}P _{p, t, \mathrm{d} \overrightarrow{p}} = |\varphi(\overrightarrow{p},t)| ^{2} \mathrm{d} ^{3} \overrightarrow{p}
\end{equation}
% subsection Mesure de l'impulsion (end)

\subsection{Valeurs moyennes de grandeurs dépendant de l'impulsion} % (fold)
\label{sub:Valeurs moyennes de grandeurs dépendant de l'impulsion}

On obtient immédiatement : 
\begin{equation}
  \langle p \rangle(t) = \int_{- \infty}^{+ \infty} p |\varphi(p,t)| ^{2} \mathrm{d} p = \int_{- \infty}^{+ \infty} \varphi ^{*}(p, t) p \varphi(p, t) \mathrm{d}p
\label{eq:1}
\end{equation}

\begin{itemize}

    \item Onde de de Broglie : $\langle p \rangle = p_0$ 
    \item Plus généralement, pour une onde $g(p)$ : 
      \begin{equation}
        \langle g \rangle = \int_{- \infty}^{+ \infty} g(p) |\varphi(p,t)| ^{2} \mathrm{d}p, \; \Delta p = \sqrt{\langle p ^{2} \rangle - \langle p \rangle ^{2}}
      \end{equation}

\end{itemize}
% subsection Valeurs moyennes de grandeurs dépendant de l'impulsion (end)

\newpage
\section{Opérateur impulsion} % (fold)
\label{sec:Opérateur impulsion}

% section Opérateur impulsion (end)
\subsection{1 dimension} % (fold)
\label{sub:Opérateur impulsion}

\subsubsection{Quantité de mouvement $p$} % (fold)
\label{sec:Quantité de mouvement $p$}

% subsubsection Quantité de mouvement $p$ (end)
D'après 
\begin{itemize}

  \item $\varphi(p,t)$ transformée de Fourier de $\psi(x,t)$.
  \item Les équations \ref{eq:1} et \ref{eq:2} démontre que $p \varphi(p,t)$ est la transformée de Fourier de la fonction $\frac{\hbar}{i} \frac{\partial }{\partial x} \psi(x,t)$,
  \item En utilisant le théorème de Parseval-Plancherel \ref{eq:Parseval-Plancherel}
    

\end{itemize}

On obtient enfin,


\begin{equation}
  \boxed{\langle p \rangle(t) = \int_{- \infty}^{+ \infty} \varphi ^{*}(p,t) \times p \varphi(p,t) \mathrm{d}p = \int_{- \infty}^{+ \infty} \psi ^{*}(x, t) \frac{\hbar}{i}  \frac{\partial }{\partial x} \psi (x,t) \mathrm{d}x}
\end{equation}

On définit :
\begin{Definition}[colbacktitle=red!75!black]{Opérateur impulsion}{}
\begin{equation}
  \hat{p} = - i \hbar \frac{\partial }{\partial x} 
  \label{eq:Opérateur impulsion}
\end{equation}
défini par son action sur une fonction $\psi(x,t)$ quelconque :
\begin{equation}
  (\hat{p}\psi)(x, t) = - i \hbar \frac{\partial \psi}{\partial x} 
\end{equation}
\end{Definition}

Nous avons simplement : 


\begin{equation}
  \boxed{\langle p \rangle(t) = \int_{- \infty}^{+ \infty} \psi ^{*}(x,t) (\hat{p} \psi)(x,t) \mathrm{d} x} 
\end{equation}


Conclusion : La quantité de mouvement $p$ comporte comme un \underline{opérateur} dans la représentation en position.

\begin{note}{}{}
  En général, chaque grandeur physique correspond un opérateur en mécanique quantique.
\end{note}


\subsubsection{Fonction $g(p)$ développable en série} % (fold)
\label{sec:Fonction $g(p)$ développable en série}

Pour une fonction 
\begin{equation}
  g(p) = \sum_{n}^{} g_n p ^{n}
\end{equation}

Donc, la valeur moyenne de $g$ associée à $\psi(x,t)$ par 
\begin{equation}
  \langle g \rangle = \sum_{n}^{} g_n \int_{- \infty}^{+ \infty} \psi ^{*}(x, t) ( \hat{p} ^{n} \psi) (x,t) \mathrm{d}x
\end{equation}

\begin{myproof}{}{} D'après l'équation \ref{eq:2} :
\begin{align}
  \langle p ^{n} \rangle(t) &= \int_{- \infty}^{+ \infty} \psi ^{*}(p, t) p ^{n} \psi(x,t) \\ 
                            &= \int_{- \infty}^{+ \infty} \left( \frac{\hbar}{i}  \right) ^{n} \psi(x,t)
\end{align}
\end{myproof}


En introduisant 
\begin{equation}
g(p) = \sum_{n}^{} g_n p ^{n} \implies  \hat{g} = \sum_{n}^{} g_n \hat{p} ^{n}
\end{equation}

On a encore 
\begin{equation}
  \hat{g} = \int_{- \infty}^{+ \infty} \psi^*(x,t) (\hat{g} \psi)(x,t) \mathrm{d}x
\end{equation}
% subsubsection Fonction $g(p)$ développable en série (end)


\subsection{3 dimension} % (fold)
\label{sub:Opérateur impulsion à 3 dimension}

Dans les trois directions cartésiennes 
\begin{equation}
  \hat{p}_x = - i \hbar \frac{\partial }{\partial x}, \;
  \hat{p}_y = - i \hbar \frac{\partial }{\partial y}, \;
  \hat{p}_z = - i \hbar \frac{\partial }{\partial z}
\end{equation}


\begin{Definition}[colbacktitle=red!75!black]{Opérateur quantité de mouvement (opérateur impulsion)}{}
\begin{equation}
  \hat{\overrightarrow{p}} = - i \hbar \; \overrightarrow{\mathrm{grad}}
\end{equation}
\end{Definition}

On peut montrer que 
\begin{equation}
  \hat{p} ^{2} = \hat{p}_x ^{2} + \hat{p}_y ^{2} + \hat{p}_z ^{2} = - \hbar ^{2} \Delta
\end{equation}

\subsection{Cas d'ondes de de Broglie} % (fold)
\label{sub:Cas d'ondes de de Broglie}

À une dimension, pour $$\psi _{p_0}(x,t) = A \exp \frac{i}{\hbar} ({p_0} . x - Et)$$

En utilisant l'expression de \ref{eq:Opérateur impulsion} : 
\begin{equation}
  \hat{p} \psi _{p_0} = p_0 \psi _{p_0}
\end{equation}

\begin{Definition}[colbacktitle=red!75!black]{État propre de l'opérateur impulsion, valeur propre}{}
En ce sens, nous disons que une onde de de Broglie est un \textbf{état propre} de l'opérateur impulsion, avec la \textbf{valeur propre} $\overrightarrow{p_0}$ la quantité de mouvement associé.

\end{Definition}



\subsection{Notation de Dirac} % (fold)
\label{sub:Notation de Dirac}

Rappel : 
\begin{equation}
  \langle \psi_1 | \psi_2 \rangle = \int_{- \infty}^{+ \infty} \psi_1 ^{*}(x,t) \psi_2(x,t) \mathrm{d}x
\end{equation}

On introduit de façon, pour un opérateur $\hat{A}$ linéaire agissant sur les fonctions d'onde : 
\begin{equation}
  \boxed{\langle \psi_1 | \hat{A} | \psi_2 \rangle = \int_{-\infty}^{+\infty} \psi_1 ^{*}(x,t) (\hat{A} \psi_2)(x,t) \mathrm{d}x}
\end{equation}

Par exemple, on a déjà vu : 
\begin{equation}
\begin{cases}
 \langle \psi_1 | \hat{p} | \psi_2 \rangle = \int_{-\infty}^{+\infty} \psi_1 ^{*}(x,t) (\hat{p} \psi_2)(x,t) \mathrm{d}x = \langle p \rangle  \\
\langle \psi_1 | \hat{x} | \psi_2 \rangle = \int_{-\infty}^{+\infty} \psi_1 ^{*}(x,t) (\hat{x} \psi_2)(x,t) \mathrm{d}x = \int_{-\infty}^{+\infty} \psi_1 ^{*}(x,t) ({x} \psi_2)(x,t) \mathrm{d}x = \langle x \rangle
\end{cases}
\end{equation}

\begin{Definition}[colbacktitle=red!75!black]{Opérateur position}{}
Nous introduisons l'\textbf{opérateur position} agissant sur les fonctions d'onde : 
\begin{equation}
  \hat{x} \psi (x,t) = x \psi(x,t)
\end{equation}
\end{Definition}

\subsection{Commutateur} % (fold)
\label{sub:Commutateur}

En général, les opérateurs sont non commutable. 
\begin{Definition}[colbacktitle=red!75!black]{Commutateur de deux opérateurs}{}
Nous le définissons comme 
\begin{equation}
  [ \hat{A},\hat{B}] = \hat{A} \hat{B} - \hat{B} \hat{A}
\end{equation}
\end{Definition}


Si deux opérateurs commutent nous avons 
\begin{equation}
  [ \hat{A}, \hat{B}] = 0
\end{equation}

\subsubsection{Exemple} % (fold)
\label{sec:Exemple}

Les deux opérateurs $\hat{x}$ et $\hat{p}_x$ ne commutent pas, en effet ($\hat{I}$ l'opérateur identité)
\begin{equation}
  [\hat{x}, \hat{p}_x] = i \hbar \hat{I}
\end{equation}

Pour les autre dimensions, on obtient :
\begin{equation}
  [ \hat{x}, \hat{p}_y] = [\hat{y}, \hat{p}_z] = \dots = 0
\end{equation}

\begin{myproof}{}{} Calcul direct : 
\begin{align}
  [\hat{x}, \hat{p}_x]f &= (\hat{x}\hat{p}_x - \hat{p}_x \hat{x}) f \\ 
                        &= x \left( -i \hbar \frac{\partial f}{\partial x}  \right) - \left(- i \hbar \frac{\partial }{\partial x} (xf)\right) \\ 
                        &= -i \hbar x \frac{\partial f}{\partial x}  + i \hbar f \times 1 + i \hbar x \frac{\partial f}{\partial x}
\end{align}
\end{myproof}


% subsubsection Exemple (end)
% subsection Commutateur (end)

% subsection Notation de Dirac (end)



% subsection Résumé (end)











\section{Principe de correspondance} % (fold)
\label{sec:Principe de correspondance}

\subsection{Relation entre la position et l'impulsion} % (fold)

La position $x$ et l'impulsion $p$ correspond chaqune un opérateur 
\begin{itemize}

    \item Dans la représentation en position $\psi(x,t)$ : 
      \begin{equation}
        \hat{x} = x, \; \hat{p} = - i \hbar \frac{\partial }{\partial x} 
      \end{equation}

    \item Dans la représentation en impulsion $\varphi(p,t)$ : (admis) 
      \begin{equation}
        \hat{p} = p, \; \hat{x} = - i \hbar \frac{\partial }{\partial p} 
      \end{equation}

\end{itemize}
% section Principe de correspondance (end)

\subsection{Principe de correspondance} % (fold)
\label{sub:Principe de correspondance}

\begin{Theorem}{Principe de correspondance}{}
Toute grandeur physique $A = f(x,p)$ est assoicée à un opérateur $\hat{A}$ linéaire. $\hat{A}$ est obtenue en remplaçant : 
\begin{itemize}

    \item $x \to \hat{x}$ 
    \item $p \to -i \hbar \frac{\partial }{\partial x} $

\end{itemize}
\end{Theorem}



\subsubsection{Exemples} % (fold)
\label{sec:Exemples}

\begin{itemize}

    \item $V(x) \to \hat{V}$ tel que $\hat{V} \psi = V \psi$
    \item Fonction polynomiale de $p$ : 
      \begin{equation}
        A(p) = \sum_{n}^{}c_n p ^{n} \implies \hat{A} \psi(p,t) = \sum_{n}^{} c_n (-i \hbar) ^{n} \frac{\partial ^{n} \psi}{\partial x ^{n}}  (x,t)
      \end{equation} 
    \item Fonction développables en séries entières : 
      \begin{equation}
        A(p) = \exp \left( - \frac{ip}{\hbar} x_0 \right) \implies (\hat{A} \psi)(x,t) = \sum_{n=0}^{\infty} \frac{1}{n!}  (-x_0) ^{n} \frac{\partial ^{n}(x,t)}{\partial x ^{n}}  = \psi(x-x_0, t)
      \end{equation}

      \begin{myproof}{}{}
      Réviser le cours \textbf{Séries Entières}.
      \end{myproof}

      


      On la note l'\textbf{opérateur de translation de $x_0$}

\end{itemize}

% subsubsection Exemples (end)

\subsection{Application en mécanique} % (fold)
\label{sub: Application en mécanique}

\begin{figure}[H] %h:当前位置, t:顶部, b:底部, p:浮动页
  \centering
  \includegraphics[width=0.9\textwidth]{./assets/Opérateurs utiles.png}
  \caption{Opérateurs utiles}
  \label{fig:Opérateurs utiles}
\end{figure}


% subsubsection Application en mécanique (end)

Remarque : \textbf{opérateur hamiltonien} : $\hat{H}$ 




% subsection Principe de correspondance (end)
