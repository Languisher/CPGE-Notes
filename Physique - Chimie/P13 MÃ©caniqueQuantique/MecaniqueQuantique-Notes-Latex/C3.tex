\chapter{Relation de Heisenberg} % (fold)
\label{chap:Relation de Heisenberg}

\begin{tcolorbox}
Relation de Heisenberg

\begin{itemize}

    \item Relation de Heisenberg position-impulsion 
    \item La critère du régime classique et quantique 
    \item Justifier la stabilité des atomes par la relation de Heisenberg 
    \item Relation de Heisenberg temps-énergie

\end{itemize}
\end{tcolorbox}

\section{Relation d'incertitude de Heisenberg (Inégalité spectrale)} % (fold)
\label{sec:Relation d'incertitude de Heisenberg (Inégalité spectrale)}

À chaque instant, on définit 
\begin{itemize}

    \item l'extension spatial du paquet d'ondes $\Delta x (t) = \sqrt{ \langle x ^{2} \rangle - \langle x \rangle ^{2}}$, adaptée à fonction $\psi(x,t)$
    \item la largeur en impulsion du même paquet $\Delta p (t) = \sqrt{ \langle p ^{2} \rangle - \langle p \rangle ^{2}}$, adaptée à fonction $\varphi(p,t)$

\end{itemize}

(Par exemple : $\langle x ^{2} \rangle = \int_{}^{} x ^{2} | \psi(x,t) | ^{2} \mathrm{d}x$)

\subsection{Inégalité spectrale} % (fold)

% section Inégalité spectrale (end)
À chaque instant, elle sont liées par l'\textbf{inégalité spectrale} : 
\begin{equation}
  \boxed{\Delta x(t) \times \Delta p(t) \ge \frac{\hbar}{2} }
\end{equation}

\subsection{Remarques} % (fold)
\label{sub:Remarques}

% subsection Remarques (end)
Points à noter :
\begin{itemize}
\item Égalité est pris lorsque les la fonction $\psi(x)$ est sous la forme \underline{gaussienne}
  \begin{equation}
    \psi(x) \propto \exp \left( - \frac{x ^{2}}{a ^{2}}  \right)
  \end{equation}
    \item Relation est relié avec le fait que $[\hat{x}, \hat{p}] \ne 0$. (Voir chapitre suivant) choisissons une autre couple pour que elles sont commutable, donc on n'a aucune borne inférieure non nulle :
      \begin{equation}
        \Delta x \Delta p_y \ge 0
      \end{equation}

    \item À trois dimensions, de la même façon 
      \begin{equation}
        \boxed{ \Delta x \times \Delta p_x \ge \frac{\hbar}{2}, \; \Delta y \times \Delta p_y \ge \frac{\hbar}{2} , \Delta z \times \Delta p_z \ge \frac{\hbar}{2} }
      \end{equation}

    \item Pour les paquet d'ondes relativement bien localisées : En ordre de grandeur, 
      \begin{equation}
        \Delta x \times \Delta p \approxeq \hbar
      \end{equation}

\end{itemize}

Conclusion évident : 
\begin{itemize}

    \item Plus la position de la particule est connue avec certitude, moins cela est le cas de son impulsion 
    \item On ne peut connaître avec certitude, la position et l'impulsion de la particule en même temps 
    \item E.g. Les états parfaitement déterminés en position sont totalement indéterminés en impulsion, elle sont des \textbf{grandeurs incompatibles}

\end{itemize}
% section Relation d'incertitude de Heisenberg (Inégalité spectrale) (end)

\section{Régime classique et quantique} % (fold)


\subsection{Exemple macroscopique} % (fold)

Une goutte d'eau : \textbf{Régime classique}
\begin{itemize}

    \item On veut que l'indétermination quantique sur la position soit plus faible que $1 \mu\mathrm{m}$. 
    \item L'incertitude en vitesse est de l'ordre de :
      \begin{equation}
        \Delta v = \frac{\Delta p}{ m}  \approx \frac{\hbar}{m \Delta x}  \approxeq 10 ^{-22} \mathrm{m}. \mathrm{s} ^{-1}
      \end{equation}
    \item Longueur de de Broglie : 
      \begin{equation}
        \lambda = \frac{h}{p}  \ll  \frac{h}{\Delta p} \ll \Delta x \ll L
      \end{equation}
    \item La caractère probabiliste est totalement négligée.

\end{itemize}
% subsection Échelle macroscopqie (end)

\subsection{Exemple microscopique} % (fold)
\label{sub:Exemple microscopique}


Électron : \textbf{Régime quantique}
\begin{itemize}

    \item $m \approxeq 10 ^{-30} \mathrm{kg}$, dans l'atome d'hydrogène, l'électron est confiné dans une sphère de rayon proche du rayon de Bohr $a_0 = 53 \mathrm{pm}$, et $\Delta x \approx a_0$

    \item Effet quantique importante : 
      \begin{equation}
        \Delta v = \frac{\Delta p}{m}  \approx \frac{\hbar}{m \Delta x}  = 2 \times 10 ^{6} \mathrm{m}. \mathrm{s} ^{-1}
      \end{equation}

    \item Longueur de de Broglie comparable avec la dimension caractéristique du système : 
      \begin{equation}
        \frac{p}{m}  \approx \frac{\hbar}{m a_0}  \implies a_0 \approx \lambda = \frac{h}{p} 
      \end{equation}



\end{itemize}

% subsection Exemple microscopique (end)


% section Exemples (end)

\newpage
\section{Stabilité des atomes} % (fold)
\textbf{Résultat observé} : Si l'électron tome sur le noyaux : $\Delta x$ très petit, donc $\Delta p$ suffisament grand pour s'échapper.

La relation de Heisenberg permet de montrer que l'énergie de l'atome est nécessairement bornée inférieurement par une valeur non nulle. 

\begin{itemize}

    \item L'énergie totale : Énergie cinétique + Énergie potentielle électrostatique 
      \begin{equation}
        E = \frac{p ^{2}}{2m}  - \frac{e ^{2}}{4 \pi \varepsilon_0 r} 
      \end{equation}

    \item Modèle : un état possédant la symétrie : (Symétrie sphérique impose $\langle p_x \rangle= \langle x \rangle = 0$)
      \begin{equation}
        \langle p ^{2} \rangle = \langle p_x ^{2}\rangle + \langle p_y ^{2} \rangle + \langle p_z ^{2} \rangle = 3 \langle p_x ^{2} \rangle = 3 \Delta p_x ^{2},\; \langle r ^{2} \rangle = 3 \Delta x ^{2}
      \end{equation}


    \item On admet que, en ordre de grandeur : 
      \begin{equation}
        \langle \frac{1}{r}   \rangle = \langle r ^{2} \rangle ^{-1 /2} 
      \end{equation}

    \item Nous aurons donc 
      \begin{align}
        \langle E \rangle &= \frac{3 \Delta p _x ^{2}}{2m}  - \frac{
          e ^{2}
        }{4 \pi \varepsilon_0 \sqrt{3} \Delta x} \\ 
                          &\ge \frac{3 \Delta p_x ^{2}}{2m}  - \frac{e ^{2} 2 \Delta p_x}{4 \pi \varepsilon_0 \sqrt{3} \hbar}  = E _{min,H}( \Delta p_x)
      \end{align}

    \item En fonction de $\Delta p_x$, la valeur moyenne de l'énergie passe par un minimum tel que 
      \begin{equation}
        \frac{\mathrm{d} E _{min, H}}{\mathrm{d} \Delta p_x}  = \frac{3 \Delta p _{x, min}}{m}  - \frac{2 e ^{2}}{ 4 \pi \varepsilon_0 \sqrt{3} \hbar}  \approxeq 1,3 a_0
      \end{equation}
      avec $a_0 = \frac{\hbar ^{2} \times 4 \pi \varepsilon_0}{ m e ^{2}}$ \textbf{le rayon de Bohr}

    \item L'énergie minimale possible est 
      \begin{equation}
        E_0 = - \frac{4}{18} \frac{\hbar ^{2}}{ma_0 ^{2}} = - \frac{4}{9} R_H
      \end{equation}
      où $R_H$ la \textbf{constante de Rydberg} $= 13,6 \mathrm{eV}$
\end{itemize}

% subsection Stabilité des atomes (end)

\newpage
\section{Relation de Heisenberg temps-énergie} % (fold)
\label{sec:Relation de Heisenberg temps-énergie}

Comme onn peut réaliser une transformation de Fourier en temps de la fonction d'onde : 
\begin{equation}
  \psi(x,t) = \frac{1}{\sqrt{2 \pi \hbar}}  \int_{- \infty}^{+ \infty} \underline{\psi}(x,E) \exp \left( - \frac{iEt}{\hbar}  \right) \mathrm{d}E \underset{T.F.}{\iff}
\underline{\psi}(x,E) = \frac{1}{\sqrt{2 \pi \hbar}}  \int_{- \infty}^{+ \infty}  \psi    (x,t) \exp \left(  \frac{iEt}{\hbar}  \right) \mathrm{d}t \end{equation}
% section Relation de Heisenberg temps-énergie (end)

L'inégalité conduit alors 
\begin{equation}
  \Delta E \times \Delta t \ge \frac{\hbar}{2} 
\end{equation}

Conclusion : La durée de vie est infinie (ou une énergie est connue), si l'atome est dans un niveau stationnaire.

On observe des \textbf{niveau fondamental}, comme seuls les niveaux ne pouvant pas se désexciter radiativement ont une énergie \underline{parfaitement définie}. Sinon 
\begin{equation}
  \Delta E \approx \frac{\hbar}{\tau} 
\end{equation}

% chapter Relation de Heisenberg (end)
