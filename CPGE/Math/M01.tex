\chapter{Bases}

\section{Majorant et maximum}

\dfn{Majorant et Minorant}{
\begin{itemize}
    \item 
Le réel $a$ est un \textbf{majorant} de $X$ si
\[
\forall x \in X, x \leq a
\]

\item Le réel $a$ est un \textbf{minorant} de $X$ si 
    \[
    \forall x \in X, x \geq a
    \]
\end{itemize}

La partie $X$ est \textbf{majorée} (resp. \textbf{minorée}) si elle admet un majorant (resp. minorant).
}

\nt{$X$ est majorée se traduit par $$\exists a \in \mathbb{R} \;  \forall x \in X \; x\le a$$}
\ex{}{Majorant de $[0,1]$ et  $[0,1[$ est  $[1, + \infty[$}

\dfn{Plus grand, plus petit élément}{Soit $a$ un élément de $X$,
\begin{itemize}
    \item $a$ est \textbf{le plus grand élément (maximum)} de $X$ si
        \[
        \forall x \in X, x \leq a
        \]

    \item $a$ est \textbf{le plus petit élément (minimum)} de $X$ si
        \[
        \forall x \in X, x \geq a
        \]
\end{itemize}    
}

\section{Fonctions réelles}



