%%% Ne pas modifier jusqu'à la ligne 25
\documentclass[a4paper,12pt]{article}
\usepackage[utf8]{inputenc}
\usepackage[T1]{fontenc} 
\usepackage[french]{babel}
\usepackage[left=2cm,right=2cm,top=3cm,bottom=2cm, headheight=1.5cm,headsep=1.5cm]{geometry}
\usepackage{lmodern}
\usepackage{lastpage}

\usepackage{amsmath,amsfonts,amssymb,dsfont}
\usepackage{graphicx}
\usepackage{enumitem}		%\enumerate-resume
\usepackage[colorlinks=true,unicode={true},hyperindex=false, linkcolor=blue, urlcolor=blue]{hyperref}

\addto\captionsfrench{\def\tablename{Tableau}}  %légendes des tableaux
%\renewcommand\thesection{\Roman{section}~-~} 
%\renewcommand\thesubsection{\Roman{section}.\Alph{subsection}~-~} 
%\renewcommand\thesubsubsection{\Roman{section}.\Alph{subsection}.\arabic{subsubsection}~-~} 

\setcounter{secnumdepth}{3}
\parindent=0pt

\usepackage{fancyhdr}
\pagestyle{fancy}

\lhead{SJTU-ParisTech} 
%%%%%%%%%%%%%%%%%%%%%%%%%%%%%%%%%%
\chead{Titre}
\rhead{Noms}
\cfoot{\thepage/\pageref*{LastPage}}

\begin{document}

\renewcommand{\labelitemi}{$\blacktriangleright$}
\renewcommand{\labelitemii}{$\bullet$}

%%% Page de garde
\thispagestyle{empty}
\begin{center}
\includegraphics[scale=0.125]{Image/SPEITlogo.png}
\end{center}
%
\vfill
%
\begin{center}
\scalebox{1.3}{\Huge{\textbf{Nom Projet}}}
\end{center}
%
\vfill\Large
%
\begin{center}
\textbf{Étudiant 1, étudiant 2, étudiant 3, étudiant 4}
\end{center}
%
\vfill
%
\begin{center}
% \includegraphics[scale=0.375]{Image/sjtu1} 
\includegraphics[scale=1.5]{Image/sjtu2}
% 两幅图选择一个
\end{center}
%
\vfill
%
\begin{center}
\textbf{Projet du semestre d'été - 2021} 
\end{center}

\clearpage
\thispagestyle{empty}\normalsize
\tableofcontents
\clearpage


\section{Introduction à \LaTeX}

Dans cette partie, nous donnons des liens permettant l'installation de \LaTeX. Puis, nous donnons quelques exemples des utilisations les plus courantes et qui devrait être suffisantes pour écrire le rapport. Néanmoins, l'aide que l'on peut trouver sur Internet est nombreuse et permet rapidement de réaliser ce que l'on souhaite.

\subsection{Installer \LaTeX}

L'installation de \LaTeX~se fait en deux étapes:
\begin{description}[align=right,labelwidth=5cm]
\item[un compilateur:] par exemple, \href{https://tug.org/texlive/}{Texlive} qui est gratuit et multi-plateforme;
\item[un éditeur:] par exemple, \href{http://www.xm1math.net/texmaker/}{Texmaker} qui est gratuit et multi-plateforme.
\end{description}

\subsection{Exemples}

\subsubsection{Quelques exemples de typographie}

Texte \underline{sous-ligné}, \textbf{en gras} ou en \textit{en italique}.

\subsubsection{Quelques exemples de présentation}

\begin{flushleft} texte à gauche \dots \end{flushleft}

\begin{center} \dots texte centré \dots \end{center}

\begin{flushright} \dots texte à droite \end{flushright}


\textbf{Une liste numérotée:}
\begin{enumerate}
\item une liste numérotée
\item avec plusieurs items
\begin{enumerate}
\item et plusieurs niveaux
\item \dots
\end{enumerate}
\item \dots \dots
\end{enumerate}

\medskip

\textbf{Une liste à puces:}
\begin{itemize}
\item une liste à puces
\item avec plusieurs items
\begin{itemize}
\item et plusieurs niveaux
\item \dots
\end{itemize}
\item \dots
\end{itemize}

% saut de page
\pagebreak

\textbf{Un tableau:}
\begin{center}
\begin{tabular}{|c|p{.1\linewidth}| r|}
\cline{2-3}  \multicolumn{1}{c}{} & \multicolumn{2}{|c|}{un tableau} \\
\hline Ligne 1 - Colonne 1 & Colonne 2 & Colonne 3 \\
\hline Ajustement automatique de la longueur & ou longueur fixée & Colonne 3 \\
\hline
\end{tabular}
\end{center}

\textbf{Une figure (insertion d'un fichier):}
\begin{figure}[h]
\begin{center}
\includegraphics[scale=0.5]{Image/sjtu2}
\end{center}
\caption{\label{fig1}titre}
\end{figure}

On peut alors faire référence à cette figure: d'après la figure \ref{fig1}, on a \dots

\subsubsection{Quelques exemples mathématiques}

On peut écrire des passages mathématiques dans le corps de texte, comme par exemple : $x=\pi$ ou encore $\sum_{k=1}^n k=\frac{n\, (n+1)}{2}$. On peut aussi écrire des équations sur une ligne séparée :
\[ \sum_{k=1}^n k=\frac{n\, (n+1)}{2}\]
ou sur plusieurs lignes :
\begin{align*}
\left| \int_a^b (f + g) \right| &=  \left| \int_a^b f +  \int_a^b g \right| \\
& \leq \left| \int_a^b f  \right| +  \left| \int_a^b g \right| \\
& \leq \int_a^b |f| +  \int_a^b |g|
\end{align*}
Il est parfois utile de numéroter les équations pour pouvoir s'y référer, par exemple:
\begin{equation}\label{eq1}
\lim_{t\to 1^+} f(t\, x)= f(x)
\end{equation}
D'après \eqref{eq1}, on a \dots 




\section{Titre de partie}

blablabla

\subsection{Titre de sous-partie}

blablabla




\end{document}