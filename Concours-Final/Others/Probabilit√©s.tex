\chapter{Variables aléatoires discrètes} % (fold)
\label{chap: Variables aléatoires discrètes}

\section{Définition} % (fold)
\label{sec:Définition}

\begin{Definition}[colbacktitle=red!75!black]{Variable aléatoire discrète}{}
On appelle \textbf{variable aléatoire discrète} sur l'espace probabilisé $\Omega$ et à valeurs dans $E$ toute \underline{application} $X : \Omega \to E$ vérifiant 
\begin{itemize}

  \item $\{X(\Omega)\}$ est fini ou dénombrable 
  \item $\forall x \in X(\Omega)$, $X ^{-1}(\{x\})$ est élément de la tribu $\mathcal{A}$.

\end{itemize}

\end{Definition}

\begin{note}{}{}
Une \textbf{variable aléatoire discrète} est une \underline{fonction} parfaitement déterminée. Ce sont les valeurs de $X$ qui vont varier.
\center 
$X$ : {Événement} $\to$ {Résultat}
\end{note}

\begin{Definition}[colbacktitle=red!75!black]{Événements valeurs}{}
Soit $X : \Omega \to E$. 

On note pour tout $x \in E$,
\begin{equation}
(X =x) = X ^{-1}(\{x\}) = \{ \omega \in \Omega, \; X(\omega) = x \}
\end{equation}
Pour tout partie $A$ de $E$,
\begin{equation}
  (X \in A) = X ^{-1}(A) = \{ \omega \in \Omega, \; X(\omega)\in A\}
\end{equation}
Il s'agit d'un événement, et l'on peut en calculer la probabilité : $P(X=x)$

\end{Definition}


\begin{Prop}{}{}
\begin{equation}
  (X \in A) = \bigcup _{x \in X(\omega) \cap A} (X=x)
\end{equation}
\end{Prop}


\begin{Definition}[colbacktitle=red!75!black]{}{}
Si $X$ une variable aléatoire discrète \textit{réelle}, $a \in \mathbb{R}$, 
\begin{equation}
  (X \le a) = X ^{-1}(]- \infty
  , a]) = \{ \omega \in \Omega, \; X(\omega) \le a\}
\end{equation}
\end{Definition}







\section{Lois} % (fold)

\begin{Definition}[colbacktitle=red!75!black]{Loi d'une variable aléatoire discrète}{}
\textbf{Loi} de la variable $X : \Omega \to E$ :
\begin{equation}
  \forall A \in X(\Omega), \; P_X(A) = P(X \in A)
\end{equation}

\end{Definition}

\begin{Corollary}{}{}
La loi est entièrement déterminée par les valeurs
\begin{equation}
\forall x \in P(\Omega),\;  p_x = P_X(x) = P(X = x)
\end{equation}
et 
\begin{equation}
  \sum_{x \in X(\omega)}^{} p_x = 1
\end{equation}
\end{Corollary}




% section  (end)
% subsection Définition (end)
\subsection{Loi uniforme} % (fold)
\label{sub:Loi uniforme}

\subsection{Loi de Bernoulli} % (fold)
\label{sub:Loi de Bernoulli}

\subsection{Loi binomiale} % (fold)
\label{sub:Loi binomiale}

\begin{Definition}[colbacktitle=red!75!black]{Loi binomiale}{}

\begin{equation}
  X \sim\mathcal{B}(n,p) = \left\{  X(\omega) = [\![0, n]\!], \; \forall k \in [\![0, n]\!], \; P(X= k) = \binom{n}{k} p ^{k} (1-p) ^{n-k} \right\}
\end{equation}
\end{Definition}



\section{Couples de variables aléatoires discrètes} % (fold)
\label{sub:Couples de variables aléatoires discrètes}

\section{Indépendance de variable aléatoires} % (fold)
\label{sub:Indépendance de variable aléatoires}

\section{Espérance} % (fold)
\label{sec:Espérance}

\begin{Definition}[colbacktitle=red!75!black]{Espérance}{}
On dit que $X$ admet une \textbf{espérance} si la famille $(xP(X=x)) _{x \in \Omega}$ est \underline{sommable}. 

L'\textbf{espérance} vaut : 
\begin{equation}
  E(X) = \sum_{x \in X(\Omega)}^{} x P(X = x)
\end{equation}

ne dépend que la loi de la variable $X$.
\end{Definition}

\begin{Example}{}{}
Si $X \sim \mathcal{B}(n,p)$, 
\begin{equation}
  E(X) = \sum_{k=0}^{n} k. \binom{n}{k} p ^{k} (1-p) ^{(n-k)} = np
\end{equation}
\end{Example}




\section{Variance} % (fold)
\label{sec:Variance}

% section Variance (end)
\section{Variables aléatoires à valeurs naturelles} % (fold)
\label{sec:Variables aléatoires à valeurs naturelles}

% section Variables aléatoires à valeurs naturelles (end)

% section Espérance (end)
% subsection Indépendance de variable aléatoires (end)
% subsection Couples de variables aléatoires discrètes (end)
% subsection Loi binomiale (end)
% subsection Loi de Bernoulli (end)
% subsection Loi uniforme (end)
% section Variables aléatoires discrètes (end)
% chapter Probabilité (end)
