\chapter{Probabilités sur un univers fini} % (fold)

\section{Définition} % (fold)
\label{sec:Définition}

\subsection{Expériences et événements aléatoire} % (fold)
\begin{Definition}[colbacktitle=red!75!black]{Expérience aléatoire, univers des possibles}{}
  \begin{itemize}

      \item 
Une \textbf{expérience aléatoire} est une expérience dont on ne peut prédire avec certitude le résultat. 

\item L'\underline{ensemble des résultats possibles} est appelé \textbf{univers des possibles}, noté $\Omega$.

  \end{itemize}
\end{Definition}

\begin{Definition}[colbacktitle=red!75!black]{
    Événement aléatoire
  }{}
Un \textbf{événement aléatoire} est un événement qui \underline{peut se produire ou non}. Il s'agit donc d'\underline{une partie de $\Omega$} : $A \in \mathscr{P}(\Omega)$.

\begin{itemize}

    \item 
L'événement $A$ est \textbf{réalisé} si \underline{le résultat $\omega$ de cette expérience est élément de $A$}.

\item $\Omega$ est l'événement \underline{certain}, $\emptyset$ est l'événement \underline{impossible}.

\end{itemize}
\end{Definition}


\begin{Definition}[colbacktitle=red!75!black]{}{}
\begin{itemize}

    \item Événement $A$ et $B$ 
    \item Événement $A$ ou $B$
    \item Événement $\overline{A}$, \textbf{contraire de $A$}.

\end{itemize}
\end{Definition}

\begin{Definition}[colbacktitle=red!75!black]{Événement incompatibles}{}
\begin{equation}
  A \cap B = \emptyset
\end{equation}
\end{Definition}

\begin{Definition}[colbacktitle=red!75!black]{Système complet d'événements (SCE)}{}
Une famille finie d'événements \underline{deux à deux incompatibles} et \underline{recouvrent} $\Omega$ : 
\begin{equation}
  \bigcup _{i \in I} A_i = \Omega, \; \forall (i,j) \in I ^{2}, i \ne j \implies A_i \cap A_j = \emptyset
\end{equation}
\end{Definition}

\subsection{Variable aléatoire} % (fold)
\label{sub:Variable aléatoire}

% subsection Variable aléatoire (end)
\begin{Definition}[colbacktitle=red!75!black]{Variable aléatoire}{}
\textbf{Variable aléatoire} sur $\Omega$ toute application $X: \Omega \to E$ définie sur l'univers $\Omega$ et à valeurs dans un ensemble $E$.

\begin{itemize}

  \item $X(\Omega) = \{ x_1, \dots, x_n\} \in E ^{n}$
  \item On note pour tout $x \in E$,
\begin{equation}
(X =x) = X ^{-1}(\{x\}) = \{ \omega \in \Omega, \; X(\omega) = x \}
\end{equation}
\item Pour tout partie $A$ de $E$,
\begin{equation}
  (X \in A) = X ^{-1}(A) = \{ \omega \in \Omega, \; X(\omega)\in A\}
\end{equation}

\end{itemize}
\end{Definition}


\section{Espaces probabilisés finis} % (fold)
\label{sec:Espaces probabilisés finis}

\subsection{Probabilité} % (fold)
\label{sub:Probabilité}
\begin{Definition}[colbacktitle=red!75!black]{Probabilité}{}

  Soit $\Omega$ un ensemble fini non vide. Toute application $P: \mathscr{P}(\Omega) \to [0,1]$ est appellé \textbf{probabilité} sur $\Omega$ si elle vérifie :
  \begin{itemize}

      \item $P(\Omega) = 1$ 
      \item et si $A$ et $B$ deux événements incompatibles, alors 
        \begin{equation}
          P ( A \cup B) = P(A) + P(B)
        \end{equation}

  \end{itemize}

  On dit alors $(\Omega, P)$ est un \textbf{espace probabilisé fini}. $\forall A \in \mathscr{P}(\Omega)$, $P(A) \in [0, 1]$
\end{Definition}

\subsection{Probabilité uniforme sur un ensemble fini} % (fold)
\label{sub:Probabilité uniforme sur un ensemble fini}
\begin{Definition}[colbacktitle=red!75!black]{
    Probabilité uniforme
  }{}
\begin{equation}
  P(A) = \frac{\mathrm{card}(A)}{\mathrm{card}(\Omega)}  = \frac{\text{nombre de cas favorables}}{\text{nombre de cas possibles}} 
\end{equation}
\end{Definition}

\subsection{Propriétés des probabilités finies} % (fold)
\label{sub:Propriétés des probabilités finies}

\begin{Theorem}{Formule d'additivité finie}{}
Si $(A_i)$ deux à deux incompatibles, alors 
\begin{equation}
  P \left( \bigcup _{i=1} ^{n} A_i \right) = \sum_{i=1}^{n} P(A_i)
\end{equation}
\end{Theorem}


% subsection Propriétés des probabilités finies (end)
\section{Conditionnement} % (fold)
\label{sec:Conditionnement}

\subsection{Probabilité conditionnelles} % (fold)

% subsection  (end)
\begin{Definition}[colbacktitle=red!75!black]{Probabilité conditionnelles}{}
Soit $(\Omega, P)$ espace probabilisé et $B$ un événement non négligeable. La \textbf{probabilité conditionnelle de $A \in \mathscr{P}(\Omega)$ sachant $B$} : 
\begin{equation}
  P(A | B) = P_B(A) = \frac{
    P(A \cap B)
  }{P(B)} 
\end{equation}
\end{Definition}

\begin{Prop}{}{}
$P_B$ est un probabilité sur $\Omega$
\end{Prop}


\subsection{Formules} % (fold)
\label{sub:Formules}

\begin{Corollary}{}{}
\begin{equation}
  P(A \cap B) = P(B) \times P_B(A)
\end{equation}
\end{Corollary}
\begin{myproof}{}{}
$A$, $B$ ont lieu en même temps = $B$ déjà a lieu + de plus, $A$ aura lieu
\end{myproof}

\begin{Corollary}{
    Inversion des conditionnements
  }{}
\begin{equation}
  P_B(A) = \frac{P(A) \times P_A(B)}{P(B)} 
\end{equation}
\end{Corollary}
\begin{Prop}{Formule des probabilités composées}{}
\begin{equation}
  P \left( \bigcap _{i=1} ^{n} A_i \right)  = P(A_1) \times P _{A_1} \times P(A_2) \times \dots P _{\bigcup _{i=1} ^{n-1} A_i}(A_n)
\end{equation}
\end{Prop}

\begin{Prop}{Formule des probabilités totales}{}
  Soit $(A_i) _{i \in [\![1, n]\!]}$ système complet d'événements non négligeables. Pour tout événements $B \in \mathscr{P}(\Omega)$, 
  \begin{equation}
    P (B) = \sum_{i=1}^{n}P(B \cap A_i) = \sum_{i=1}^{n} P(A_i) P _{A_i}(B)
  \end{equation}

\end{Prop}

\begin{Theorem}{\color{red} Formule de Bayes}{}

  Soit $(A_i) _{i \in [\![1, n]\!]}$ système complet d'événements non négligeables. Pour tout événements $B \in \mathscr{P}(\Omega)$, et pour tout $j \in [\![1, n]\!]$, 
  \begin{equation}
    P_B( A_j) = \frac{P (A_j) \times P _{A_j}(B)}{ \sum_{i=1}^{n} P(A_i) \times P _{A_i}(B)} 
  \end{equation}
\end{Theorem}

\begin{myproof}{}{}
Combinaison des propositions précédantes.
\end{myproof}


\section{Indépendance en probabilité} % (fold)
\label{sec:Indépendance en probabilité}

\subsection{Indépendance de deux événements} % (fold)
\label{sub:Indépendance de deux événements}
\begin{Definition}[colbacktitle=red!75!black]{Indépendants pour la probabilité $P$}{}
Deux événements $A$ et $B$ sont \textbf{indépendants pour la probabilité} $P$ lorsque
\begin{equation}
  P(A \cap B) = P(A) \times P(B)
\end{equation} 
En particulier, si $P(B) >0$, donc ils sont indépendants si et seulement si 
\begin{equation}
  P_B(A) = P(A)
\end{equation}
\end{Definition}

\subsection{Mutuellement indépendants} % (fold)
\label{sub:Mutuellement indépendants}

% subsection MUtuellement indépendants (end)

\begin{Definition}[colbacktitle=red!75!black]{Mutuellement indépendants}{}
  Pour tous $i_1, \dots, i_k \in [\![1, n]\!]$, 
  \begin{equation}
    P( A _{i_1} \cap \dots \cap A _{i_k}) = P(A _{i_1}) \times \dots P(A _{i _k})
  \end{equation}
\end{Definition}

\begin{Prop}{}{}
Mutuellement indépendant $\implies$ Deux à deux indépendant, maix la réciproque est fausse en général.
\end{Prop}





% subsection Indépendance de deux événements (end)
% subsection Indépendance en probabilité (end)









% subsection Formules (end)



% section Conditionnement (end)

% subsection  (end)

% subsection Probabilité uniforme sur un ensemble fini (end)
% subsection Probabilité (end)
% subsection Espaces probabilisés finis (end)









% subsection Expériences et événements aléatoire (end)
\begin{Definition}[colbacktitle=red!75!black]{Variable aléatoire discrète}{}
On appelle \textbf{variable aléatoire discrète} sur l'espace probabilisé $\Omega$ et à valeurs dans $E$ toute \underline{application} $X : \Omega \to E$ vérifiant 
\begin{itemize}

  \item $\{X(\Omega)\}$ est fini ou dénombrable 
  \item $\forall x \in X(\Omega)$, $X ^{-1}(\{x\})$ est élément de la tribu $\mathcal{A}$.

\end{itemize}

\end{Definition}

\begin{note}{}{}
Une \textbf{variable aléatoire discrète} est une \underline{fonction} parfaitement déterminée. Ce sont les valeurs de $X$ qui vont varier.
\center 
$X$ : {Événement} $\to$ {Résultat}
\end{note}

\begin{Definition}[colbacktitle=red!75!black]{Événements valeurs}{}
Soit $X : \Omega \to E$. 

On note pour tout $x \in E$,
\begin{equation}
(X =x) = X ^{-1}(\{x\}) = \{ \omega \in \Omega, \; X(\omega) = x \}
\end{equation}
Pour tout partie $A$ de $E$,
\begin{equation}
  (X \in A) = X ^{-1}(A) = \{ \omega \in \Omega, \; X(\omega)\in A\}
\end{equation}
Il s'agit d'un événement, et l'on peut en calculer la probabilité : $P(X=x)$

\end{Definition}


\begin{Prop}{}{}
\begin{equation}
  (X \in A) = \bigcup _{x \in X(\omega) \cap A} (X=x)
\end{equation}
\end{Prop}


\begin{Definition}[colbacktitle=red!75!black]{}{}
Si $X$ une variable aléatoire discrète \textit{réelle}, $a \in \mathbb{R}$, 
\begin{equation}
  (X \le a) = X ^{-1}(]- \infty
  , a]) = \{ \omega \in \Omega, \; X(\omega) \le a\}
\end{equation}
\end{Definition}







\section{Lois} % (fold)

\begin{Definition}[colbacktitle=red!75!black]{Loi d'une variable aléatoire discrète}{}
\textbf{Loi} de la variable $X : \Omega \to E$ :
\begin{equation}
  \forall A \in X(\Omega), \; P_X(A) = P(X \in A)
\end{equation}

\end{Definition}

\begin{Corollary}{}{}
La loi est entièrement déterminée par les valeurs
\begin{equation}
\forall x \in P(\Omega),\;  p_x = P_X(x) = P(X = x)
\end{equation}
et 
\begin{equation}
  \sum_{x \in X(\omega)}^{} p_x = 1
\end{equation}
\end{Corollary}





% section  (end)
% subsection Définition (end)
\subsection{Loi uniforme} % (fold)
\label{sub:Loi uniforme}

\subsection{Loi de Bernoulli} % (fold)
\label{sub:Loi de Bernoulli}

\subsection{Loi binomiale} % (fold)
\label{sub:Loi binomiale}

\begin{Definition}[colbacktitle=red!75!black]{Loi binomiale}{}

\begin{equation}
  X \sim\mathcal{B}(n,p) = \left\{  X(\omega) = [\![0, n]\!], \; \forall k \in [\![0, n]\!], \; P(X= k) = \binom{n}{k} p ^{k} (1-p) ^{n-k} \right\}
\end{equation}
\end{Definition}



\section{Couples de variables aléatoires discrètes} % (fold)
\label{sub:Couples de variables aléatoires discrètes}

\section{Indépendance de variable aléatoires} % (fold)
\label{sub:Indépendance de variable aléatoires}

\section{Espérance} % (fold)
\label{sec:Espérance}

\begin{Definition}[colbacktitle=red!75!black]{Espérance}{}
On dit que $X$ admet une \textbf{espérance} si la famille $(xP(X=x)) _{x \in \Omega}$ est \underline{sommable}. 

L'\textbf{espérance} vaut : 
\begin{equation}
  E(X) = \sum_{x \in X(\Omega)}^{} x P(X = x)
\end{equation}

ne dépend que la loi de la variable $X$.
\end{Definition}

\begin{Example}{}{}
Si $X \sim \mathcal{B}(n,p)$, 
\begin{equation}
  E(X) = \sum_{k=0}^{n} k. \binom{n}{k} p ^{k} (1-p) ^{(n-k)} = np
\end{equation}
\end{Example}




\section{Variance} % (fold)
\label{sec:Variance}

% section Variance (end)
\section{Variables aléatoires à valeurs naturelles} % (fold)
\label{sec:Variables aléatoires à valeurs naturelles}

% section Variables aléatoires à valeurs naturelles (end)

% section Espérance (end)
% subsection Indépendance de variable aléatoires (end)
% subsection Couples de variables aléatoires discrètes (end)
% subsection Loi binomiale (end)
% subsection Loi de Bernoulli (end)
% subsection Loi uniforme (end)
% section Variables aléatoires discrètes (end)
% chapter Probabilité (end)
