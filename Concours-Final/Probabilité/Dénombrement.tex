\chapter{Dénombrement} % (fold)
\label{chap:Dénombrement}

\section{Cardinal d'un ensemble fini} % (fold)
\label{sec:Cardinal d'un ensemble fini}

\subsection{Cardinal d'un ensemble fini} % (fold)
\label{sub:Cardinal d'un ensemble fini}

\begin{Definition}[colbacktitle=red!75!black]{
    Ensemble fini, cardinal d'un ensemble fini
  }{}

  $E$ un ensemble non vide est dit \textbf{ensemble fini} s'il existe $n \in \mathbb{N} ^{*}$ et une \underline{bijection} de $E$ dans $[\![1, n]\!]$. 

  $n$ est unique et appelé \textbf{cardinal} de $E$ : $\boxed{\mathrm{Card}(E)}$.

\end{Definition}

% subsection Cardinal d'un ensemble fini (end)

\subsection{Cardinal d'une partie} % (fold)
\label{sub:Cardinal d'une partie}

\begin{Theorem}{Cardinal des parties}{}

Soit $E$ un ensemble fini et $A$ une partie de $E$. Alors, 
\begin{itemize}

    \item $A$ est un \textbf{ensemble fini} et $\mathrm{card}(A) \le \mathrm{card}(E)$ 
    \item $A =E \iff \mathrm{card}(A) = \mathrm{card}(E)$

\end{itemize}
\end{Theorem}


% subsection Cardinal d'une partie (end)

\subsection{Application entre deux ensembles} % (fold)
\label{sub:Application entre deux ensembles}

\begin{Prop}{}{}
Soit $f$ une application de $E$ dans $F$, où $E$ et $F$ sont deux ensembles finis. 
Si (il 'existe) $f$ (qui) est 
\begin{itemize}

    \item injective, $\mathrm{card}(A) \le \mathrm{card}(E)$ 
    \item surjective, $\mathrm{card}(A) \ge \mathrm{card}(E)$
    \item bijective, $\mathrm{card}(A) = \mathrm{card}(E)$

\end{itemize}
\end{Prop}

\begin{Theorem}{}{}
Soit $E$ et $F$ deux ensembles finis \underline{de même cardinal}, et $f$ une application de $E$ et $F$. Les propriétés suivantes sont équivalentes : 
\begin{itemize}

    \item $f$ est injective
    \item $f$ est surjective 
    \item $f$ est bijective

\end{itemize}
\end{Theorem}


% subsection Application entre deux ensembles (end)

\subsection{Opérations sur les cardinaux} % (fold)
\label{sub:Opérations sur les cardinaux}

\begin{Theorem}{Produit cartésien de deux ensembles finis}{}
\begin{equation}
  \mathrm{card}(E \times F ) = \mathrm{card}(E) \times \mathrm{card}(F)
\end{equation}
\end{Theorem}

\begin{Theorem}{Réunion de deux ensembles finis}{}
\begin{equation}
  \mathrm{card}(E \cup F) = \mathrm{card}(E) + \mathrm{card}(F) - \mathrm{card}(E \cap F)
\end{equation}

De plus, si les deux ensembles sont disjoints, alors 
\begin{equation}
  \mathrm{card}(E \cup F ) = \mathrm{card}(E) + \mathrm{card}(F)
\end{equation}
\end{Theorem}

\begin{Corollary}{Complémentaire}{}
\begin{equation}
  \mathrm{card}(C_EA) = \mathrm{card}(E) - \mathrm{card}(A)
\end{equation}
\end{Corollary}

\subsection{Cardinal de l'ensemble des applications d'un ensemble fini dans un autre} % (fold)
\label{sub:Cardinal de l'ensemble des applications d'un ensemble fini dans un autre}
\begin{Theorem}{}{}

\begin{equation}
  \mathrm{card}(\mathcal{F}(E,F)) = (\mathrm{card}({\color{red} F})) ^{\mathrm{card}(\color{red} E)} \color{red}
\end{equation}

\end{Theorem}

\begin{myproof}{}{}
Pour chaque $x_i$ dans $E$, on choisit $f(x_i)$ dans $E$ donc pour chaque élément on a $\mathrm{card}(F)$ possibilités.

Cela implique que il y 
\begin{equation}
  \mathrm{card}(F) \times \mathrm{card}(F) \times \dots \times \mathrm{card}(F)
\end{equation}
possibilités.
\end{myproof}



% subsection Cardinal de l'ensemble des applications d'un ensemble fini dans un autre (end)


\subsection{Nombre de parties d'un ensemble fini} % (fold)
\label{sub:Nombre de parties d'un ensemble fini}

\begin{Theorem}{}{}

\begin{equation}
  \mathrm{card}(\mathscr{P}(E)) = 2 ^{\mathrm{card}(E)}
\end{equation}
\end{Theorem}

\begin{myproof}{}{}
  Pour une partie de $E$ : $A$, considérer la fonction $\mathbb{1} _{A} : E \to \{0, 1\}$
\end{myproof}



% subsection Nombre de parties d'un ensemble fini (end)




% subsection Opérations sur les cardinaux (end)
% section Cardinal d'un ensemble fini (end)

\section{Liste et combinaisons} % (fold)
\label{sec:Liste et combinaisons}

\subsection{$p$-listes} % (fold)
\label{sub:p-listes}

\begin{Definition}[colbacktitle=red!75!black]{$p$-liste d'élément de $E$}{}
Soit $E$ un ensemble fini. Une \textbf{$p$-liste d'éléments de $E$} est un élément de la forme $(x_1, \dots,x_p) \in E ^{p}$.
\end{Definition}

\begin{Theorem}{Nombre de $p$-listes}{}
Le nombre de $p$-listes de $E$ est $(\mathrm{card}(E)) ^{p}$. 
\end{Theorem}

\begin{myproof}{}{}
  $\mathrm{card}(E) \times \dots \times \mathrm{card}(E)$
\end{myproof}

\begin{Theorem}{Nombre de $p$-listes d'éléments distincts}{}
Soit $n = \mathrm{card}(E)$. Le nombre de $p$-listes de $E$ d'éléments \textbf{distincts} de $E$ est égal à 
\begin{equation}
  \frac{n!}{(n-p)!} = n \times (n-1) \times \dots \times (n-p+1)
\end{equation}
\end{Theorem}

\begin{Theorem}{Nombre d'injections}{}
Le nombre d'applications injectives \underline{d'un ensemble de cardinal $p$} dans \underline{unsemble de cardinal $n$} est : 
\begin{equation}
  \frac{n!}{(n-p)!} 
\end{equation}
\end{Theorem}

\begin{note}{}{}
  Application injective $\iff$ $p$-listes d'éléments distincts, tout $x_i$ correpond $y_i$ unique, ensuite il y a $p$ éléments dans l'ensemble $\{y_1, \dots, y_n\}$ selectionnés.
\end{note}















% subsection $p$-listes (end)

\subsection{Permutations} % (fold) 
\begin{Definition}[colbacktitle=red!75!black]{Permutation}{}
Une \textbf{permutation} est une bijection de $E$ dans lui-même.
\end{Definition}

\begin{Theorem}{Nombre de permutations}{}
Soit $E$ un ensemble fini de cardinal $n \in \mathbb{N} ^{*}$, 
le nombre de permutations de $E$ dans lui-même est 
\begin{equation}
  n! = n \times (n-1) \times \dots \times 1
\end{equation}
\end{Theorem}


\begin{Theorem}{Nombre de bijections}{}
\begin{equation}
  n! = n \times (n-1) \times \dots \times 1
\end{equation}
\end{Theorem}





% subsection  (end)
\subsection{$p$-combinaisons} % (fold)
\label{sub:$p$-combinaisons}

\begin{Definition}[colbacktitle=red!75!black]{$p$-combinaison}{}
Soit $E$ un ensemble fini. \textbf{$p$-combinaison} de $E$ est \underline{toute partie de $E$ à $p$ éléments}.
\end{Definition}

\begin{Theorem}{}{}
Le nombre de $p$-combinaisons de $E$ : 
\begin{equation}
  \binom{n}{p} = \begin{cases}
    \frac{n!}{p!(n-p)!} \text{ si } p \in [\![0, n]\!] \\
    0
  \end{cases}
\end{equation}
\end{Theorem}

\begin{myproof}{}{}
Récurrence : Pour $a \in E$ avec $\mathrm{card}(E) = n+1$ fixé. Nombre $p$-combinaison = Nombre de parties contenant $a$ + Nombre de parties ne contenant pas $a$

\begin{equation}
  \binom{n+1}{p} = \binom{n}{p} + \binom{n}{p-1}
\end{equation}
\end{myproof}


\begin{Theorem}{
  }{}
Nombre de parties à $p$ éléments dans un ensemble à $n$ éléments : 
\begin{equation}
  \binom{n}{p}
\end{equation}
\end{Theorem}




\subsection{Règles de calcul sur les coefficients binomiaux} % (fold)
\label{sub:Règles de calcul sur les coefficients binomiaux}

\begin{Theorem}{}{}
\begin{equation}
  \forall p \in [\![0, n]\!], \; \binom{n}{p} = \binom{n}{n-p} 
\end{equation}
\end{Theorem}

\begin{Theorem}{}{}
\begin{equation}
  \forall p \in [\!1,n]\!], \; \binom{n}{p} = \frac{n}{p} \binom{n-1}{p-1}
\end{equation}
\end{Theorem}


\begin{Theorem}{}{}
\begin{equation}
  \sum_{p=0}^{n} \binom{n}{p} = 2 ^{n}
\end{equation}
\end{Theorem}

\begin{myproof}{}{}
Tous les parties dans un ensemble de cardinal $n$
\end{myproof}




\begin{Theorem}{Formule de Pascal}{}
\begin{equation}
\forall p \in [\![1, n-1]\!], \; \binom{n}{p}= \binom{n-1}{p} +\binom{n-1}{p-1}
\end{equation}
\end{Theorem}


\begin{Theorem}{Formule du binôme de Newton}{}
\begin{equation}
  \forall (x, y) \in \mathbb{C} ^{2}, \; (x+ y) ^{n} = \sum_{p=0}^{n} \binom{n}{p} x ^{n} y ^{n-p}
\end{equation}
\end{Theorem}





% subsection Règles de calcul sur les coefficients binomiaux (end)


% subsection $p$-combinaisons (end)
% section Liste et combinaisons (end)

% section  (end)
% chapter Dénombrement (end)
