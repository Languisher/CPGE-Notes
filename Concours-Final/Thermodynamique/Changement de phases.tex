\chapter{Changement de Phase} % (fold)
\label{chap:Changement de Phase}

\section{Diagramme d'équilibre} % (fold)
\label{sec:Diagramme d'équilibre}

\begin{figure}[H] %h:当前位置, t:顶部, b:底部, p:浮动页
  \centering
  \includegraphics[width=\textwidth]{./assets/Diagramme (P,T) dans le cas général puis dans le cas particulier de l'eau.png}
  \caption{Diagramme (P,T) dans le cas général puis dans le cas particulier de l'eau}
  \label{fig:Diagramme (P,T) dans le cas général puis dans le cas particulier de l'eau}
\end{figure}

Mémoire : considérer $PV = nRT \implies \rho =PM/RT$


\section{Enthalpie} % (fold)
\label{sec:Enthalpie}

\subsection{Enthalpie de changement de phase} % (fold)
\label{sub:Enthalpie de changement de phase}

Deux phases 1 et 2 existent à $T$ et une pression $P = P _{eq}(T)$. 

On appelle \textbf{enthalpie molaire de changement de phase} (1 vers 2) : 
\begin{equation}
  \Delta _{12} H_m(T) = H _{m2} (T , P _{eq}(T)) - H _{m1}(T, P _{eq}(T))
\end{equation}

\subsubsection{Transformation monobare} % (fold)
\label{sec:Transformation monobare}
\begin{itemize}

    \item Premier principe : 
      \begin{equation}
        [H] _i ^{f} = Q _{ext}
      \end{equation}

    \item Variation d'enthalpie : 
      \begin{equation}
        [H] _i ^{f} = n \times\Delta _{12} H_m(T)
      \end{equation}

    \item Finalement,
      \begin{equation}
        \Delta _{12}H_m(T) = \frac{
          Q _{ext}
        }{n} 
      \end{equation}

\end{itemize}
% subsubsection Transformation monobare (end)
\subsubsection{Ordre} % (fold)
\label{sec:Ordre}

% subsubsection Ordre (end)
\textbf{ODG} : $\Delta _{\text{vap}}H_m >0$, $\Delta _{\text{fus}}H_m >0$, $\Delta _{sub} H_m >0$.

Si la phase 2 est plus d'esordonn'ee que la phase 1, alors $\Delta _{12}H_m >0$

% subsection Enthalpie de changement de phase (end)
% section Enthalpie (end)
% section Diagramme d'équilibre (end)

\subsection{Enthalpie d' un sysème biphasé} % (fold)
\label{sub:Enthalpie d' un sysème biphasé}

Soit un corps pur comprenant une quantité $n_1 = X_1n$ et $n_2 = X_2n$, l'enthalpie est :
\begin{equation}
  \boxed{H = n_1 \times H _{m1} + n_2 \times H _{m2}}
\end{equation}

ou encore 
\begin{equation}
  X_2 = \frac{H_m - H _{m1}}{\Delta _{12}H_m} 
\end{equation}
% subsection Enthalpie d' un sysème biphasé (end)

\subsection{Transformation générale} % (fold)
\label{sub:Transformation générale}
L'état initial : phase 1, $T_i$, $P= P _{eq}(T_i)$ à phase 2, $T_f$, $X_{2f}$, $P = P _{eq}(T_f)$ = Transformation monobare + Changement de phase à $(T_f, P = P _{eq}(T_f))$
\begin{equation}
  [H]_i ^{f} = n \times (C _{pm1} \times(T_f - T_i) + X _{2f} \times \Delta _{12}H_m(T_f))
\end{equation}

\begin{figure}[H] %h:当前位置, t:顶部, b:底部, p:浮动页
  \centering
  \includegraphics[width=0.8\textwidth]{./assets/Enthalpie - transformation générale.png}
  \caption{Enthalpie - transformation générale}
\end{figure}

Note : État initial : $T_i, X _{1i} = 1$, État intermédiaire : $T_f, X _{1i}=1$


\section{Entropie} % (fold)
\label{sec:Entropie}


\subsection{Entropie molaire de changement de phase} % (fold)
\label{sub:Entropie molaire de changement de phase}
\begin{equation}
  \Delta _{12} S_m(T) = S _{m2}(T, P _{eq}(T) ) - S _{m1} (T, P _{eq}(T))
\end{equation}
% subsection Entropie molaire de changement de phase (end)

\subsection{Relation avec l'enthalpie} % (fold)
\label{sub:Relation avec l'enthalpie}

\begin{equation}
  \Delta _{12}S_m(T) = \frac{
    \Delta _{12}H_m(T)
  }{T} 
\end{equation}
% subsection Relation avec l'enthalpie (end)
% section Entropie (end)

% subsection Transformation générale (end)
% chapter Changement de Phase (end)
