\chapter{Deuxième principe de la thermodynamique} % (fold)
\label{chap:Deuxième principe. Bilans d'entropie}

\textbf{Introduction} : Le premier principe (conservation de l’énergie) pose des limites sur les transformations thermodynamiques acceptables : pour un système isolé, une transformation de l’état (a) à l’état (b) n’est possible que si U(a) = U(b) ou, dit autrement, si $\Delta U = 0$. D’après le premier principe, si la transformation de (a) vers (b) est possible, alors celle de (b) vers (a) l’est également.

Cependant, l’expérience montre qu’il n’existe pour chaque système (et chaque choix de U, V, N, etc.) qu’un seul état d’équilibre bien déterminé, et que tout système isolé évolue spontanément et de manière irréversible vers cet état d’équilibre. Le premier principe de la thermodynamique ne suffit pas pour expliquer cette observation, et l’on a besoin d’un second principe pour déterminer l’état d’équilibre. 


\section{Entropie} % (fold)
\subsection{Macro-état et macro-état d'un système} % (fold)
\label{sub:Macro-état et macro-état d'un système}

\subsection{Définition statistique de l'entropie} % (fold)
\label{sub:Définition statistique de l'entropie}

\begin{Theorem}{Formule de Boltzmann}{}
\begin{equation}
  S = k_B \ln \Omega
\end{equation}

où $\Omega$ le nombre d'états microscopiques différents décrivant l'état macroscopique considéré.
\end{Theorem}



% subsection Définition statistique de l'entropie (end)

% subsection Macro-état et macro-état d'un système (end)

% section Entropie (end)
\label{sec:Fonction d'état entropie}


\subsection{Grandeurs d'état entropie} % (fold)
\label{sub:Grandeurs d'état entropie}


\subsubsection{Fonction d'état fondamentale} % (fold)
\label{sec:Fonction d'état fondamentale}

L'entropie est une grandeur d'état qui peut être exprimée en fonction de deux autres grandeurs. 
\begin{equation}
  (U, V) \to S(U,V) \text{ vérifiant } S = S(U,V)
\end{equation}

% subsubsection Fonction d'état fondamentale (end)

\subsubsection{Identité fondamentale} % (fold)
\label{sec:Identité fondamentale}

\begin{equation}
  \left( \frac{\partial S}{\partial U}  \right)_V = \frac{1}{T} , \quad 
  \left( \frac{\partial S}{\partial V}  \right)_U = \frac{P}{T} 
\end{equation}

Au cours d'une petite variation de $\mathrm{d}U$ et $\mathrm{d}V$, on aura 
\begin{equation}
  \mathrm{d}S = \frac{1}{T}  \mathrm{d}U + \frac{P}{T}  \mathrm{d}V = \frac{1}{T}  \mathrm{d}H - \frac{V}{T}  \mathrm{d}P
\end{equation}

\begin{Theorem}{Identités thermodynamique fondamentales}{}
Ils définissent de manière thermodynamique la \underline{pression} et la \underline{température} : 
\begin{equation}
  \boxed{\mathrm{d}U = T \mathrm{d}S - p \mathrm{d} V \implies \mathrm{d}H = T \mathrm{d} S + V \mathrm{d}p}
\end{equation}

Enfin, 
\begin{equation}
  T = \left( \frac{\partial U}{\partial S}  \right)_V, \quad p = -\left(  \frac{\partial U}{\partial V}  \right)_S
\end{equation}
\end{Theorem}

\begin{myproof}{}{}
\begin{equation}
  \mathrm{d} f(x,y) =  \frac{\partial f}{\partial x}  \mathrm{d}x + \frac{\partial f}{\partial y}  \mathrm{d}y
\end{equation}
\end{myproof}





% subsubsection Identité fondamentale (end)

% subsection Grandeurs d'état entropie (end)

\subsubsection{Paramètres pratiques} % (fold)
\label{sec:Paramètres pratiques}

\begin{equation}
  \left( \frac{\partial S}{\partial T}  \right)_V = \frac{C_V}{T} , \quad 
  \left( \frac{\partial S}{\partial T}  \right)_P = \frac{
    C_P
  }{T} 
\end{equation}

\begin{myproof}{}{}

1. 
  \begin{gather}
    S(T, V) = S(U(T, V), V)
  \end{gather}
  2.
\begin{gather}
  S= S(U, V) = S(U(P, T), V(P, T)) \\ 
  \left( \frac{\partial S}{\partial T}  \right)_P = \left( \frac{\partial S}{\partial U}  \right)_V \left( \frac{\partial U}{\partial T}  \right)_P + \left( \frac{\partial S}{\partial V}  \right)_U + \left( \frac{\partial V}{\partial T}  \right)_P = \frac{1}{T} \left( \frac{\partial (U + PV)}{\partial T}  \right)_P
\end{gather}
\end{myproof}


% subsubsection Paramètres pratiques (end)


% subsection  (end)
\subsection{Entropie d'un gaz parfait} % (fold)
\label{sub:Entropie d'un gaz parfait}

\subsubsection{Couplage $(V,T)$} % (fold)

% subsubsection Couplage $(V,T)$ (end)

Pour un gaz parfait, $C_V$ ne dépend que $T$. 

\begin{gather}
  S(T, V) = S(T_0, V_0) + nR \ln \left( \frac{V}{V_0}  \right) + \frac{nR}{\gamma - 1}  \ln \left( \frac{T}{T_0}  \right)
\end{gather}

\begin{myproof}{}{}
  \begin{align}
    \mathrm{d}S &= \frac{1}{T} \mathrm{d} T + \frac{P}{T} \mathrm{d} V \\  
                &= \frac{C_V}{T} \mathrm{d}T + \frac{nR}{V}  \mathrm{d} V
  \end{align}
  Donc, 
  \begin{equation}
    \Delta S = \int_{T}^{} C_V \ln \left( \frac{T}{T_0}  \right)+ \int_{V}^{} nR \ln\left( \frac{V}{V_0}  \right)
  \end{equation}
\end{myproof}

\subsubsection{Couplage $(P,V)$} % (fold)
\label{sec:Couplage $(P,V)$}

Pour un gaz parfait, 
\begin{equation}
  S(P,V) = S(P_0, V_0) + \frac{nR}{\gamma-1}  \left[ \ln \left( \frac{P_2}{P_1}  \right) + \gamma \ln \left( \frac{V_2}{V_1}  \right) \right]
\end{equation}

\begin{myproof}{}{}
\begin{note}{}{}
Différencier logarithmiquement l'équation d'état $PV = nRT$ : 
\begin{equation}
  \frac{\mathrm{d}p}{p}  +  \frac{\mathrm{d}V}{V} = \frac{\mathrm{d}T}{T} 
\end{equation}
\end{note}

Donc, 
\begin{align}
  \mathrm{d} S &= \frac{nR}{\gamma-1} \frac{\mathrm{d}T}{T}  + nR \frac{\mathrm{d}V}{V}  \\ 
               &= \left( \frac{nR}{\gamma-1}  \right) \left( \frac{\mathrm{d}p}{p}  + \left( \frac{\mathrm{d}V}{V}  \right) \right) + nR \frac{\mathrm{d}V}{V}  \\ 
               &= \frac{nR}{\gamma-1}  \left( \frac{\mathrm{d}p}{p}  + \gamma \frac{\mathrm{d}V}{V}  \right)
\end{align}


\end{myproof}


% subsubsection Couplage $(P,V)$ (end)


% subsection Entropie d'un gaz parfait (end)

\subsection{Entropie d'une phase condensée idéale} % (fold)
\label{sub:Entropie d'une phase condensée idéale}
Comme $V$ est presque inchangé,  le seule paramètre est la température : 
\begin{equation}
  S(T) = S(T_0) + \int_{T_0}^{T} \frac{C_V(T')}{T'}  \mathrm{d}T '
\end{equation}

Dans le cas de capacité thermique constante :
\begin{equation}
  S(T) = S(T_0) + C \ln \left( \frac{T}{T_0}  \right)
\end{equation}
% subsection Entropie d'une phase condensée idéale (end)
% section Fonction d'état entropie (end)

\newpage
\section{Deuxième principe de la thermodynamique} % (fold)
\label{sec:Deuxième principe de la thermodynamique}

\subsection{Énoncé} % (fold)
\label{sub:Énoncé}


\textbf{Principe d'évolution} : L'entropie d'un \underline{système isolé} ne peut \underline{qu}'\underline{augmenter} ou \underline{rester constante} au cours du temps.
% subsection Énoncé (end)

\subsection{Équilibre thermodynamique d'un système isolé} % (fold)
\label{sub:Équilibre thermodynamique d'un système isolé}

Un état d'équilibre d'un système isolé est un état dans lequel l'entropie est \underline{maximale}.
% subsection Équilibre thermodynamique d'un système isolé (end)

\subsubsection{Exemple} % (fold)
\label{sec:Exemple}

[fig]
On doit avoir  
\begin{equation}
  T_1 = T_2
\end{equation}

\begin{myproof}{}{}
\begin{gather}
  S = S(U_1, V_1) + S(U_2, V_2) = S(U_1, V_1) + S(U_0 - U_1, V_2) \\ 
  \frac{\mathrm{d}S}{\mathrm{d}U_1} = \frac{1}{T_1(U_1, V_1)} - \frac{1}{T_2(U_0-U_1, V_2)}  = 0
\end{gather}
\end{myproof}

Si $T_1>T_2$, comme $S$ augmente en même temps, $U_1$ ne peut que diminuer.


% subsubsection Exemple (end)


\subsection{Transformation réversible} % (fold)
\label{sub:Transformation réversible}

\subsubsection{Transformation quasi-statique} % (fold)
\label{sec:Transformation quasi-statique}

% subsubsection Transformation quasi-statique (end)

\subsubsection{Transformation isentropique} % (fold)
\label{sec:Transformation isentropique}

Exemple : Transformation adiabatique quasi-statique d'un gaz parfait. 
\begin{equation}
  \mathrm{d}U = - P \mathrm{d} V \implies \mathrm{d}S = \frac{\mathrm{d}U + P \mathrm{d}V}{T}  = 0 
\end{equation}
% subsubsection Transformation isentropique (end)

\subsubsection{Entropie créée} % (fold)
\label{sec:Entropie créée}

L'\textbf{entropie créée} nous permet de quantifier le caractère réversible ou non : 
\begin{equation}
  S _{cr} = [S] _{i} ^{f}
\end{equation}

D'après le deuxième principe : 
\begin{equation}
  S _{cr} \ge _{rev} 0
\end{equation}
% subsubsection Entropie créée (end)
% subsection Transformation réversible (end)

% section Deuxième principe de la thermodynamique (end)

\section{Bilan entropique} % (fold)
\label{sec:Bilan entropique}

\subsection{Source de chaleur} % (fold)
\label{sub:Source de chaleur}

\begin{equation}
  [S _{th}] _i ^{f} = \frac{Q _{th}}{T_0} 
\end{equation} 
% subsection Source de chaleur (end)

\subsection{Système en contact avec une source de chaleur} % (fold)
\label{sub:Système en contact avec une source de chaleur}

\begin{equation}
  [S] _{i} ^{f} = S _{cr} + S _{ech}
\end{equation}
% subsection Système en contact avec une source de chaleur (end)

\subsection{Contact avec plusieurs thermostat} % (fold)
\label{sub:Contact avec plusieurs thermostat}

\begin{equation}
  S _{ech} = \sum_{i}^{} \frac{Q_i}{T_i} 
\end{equation}
% subsection Contact avec plusieurs thermostat (end)
% section Bilan entropique (end)


% chapter Deuxième principe. Bilans d'entropie (end)
